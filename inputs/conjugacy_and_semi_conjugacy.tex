 
Conjugacy can be defined as follows \cite{gelman2013bayesian}. If $\mathcal{F}$ is a class of sampling distributions and $\mathcal{P}$ is a class of prior distributions for $\param$, then the class $\mathcal{P}$ is \textit{conjugate} for $\mathcal{F}$ if
\[  p(\param \cond y ) \in \mathcal{P} \; \text{for all} \; p(\cdot \cond \param) \in \mathcal{F} \; \text{and} \; p(\cdot) \in \mathcal{P} \]

Conditional conjugacy (sometimes called semi-conjugacy) can be defined similarly  \cite{gelman2013bayesian}.  If $\mathcal{F}$ is a class of sampling distributions and $\mathcal{P}$ is a class of prior distributions for $\param \cond \phi$, then the class $\mathcal{P}$ is \textit{conditionally conjugate} for $\mathcal{F}$ if
\[  p(\param \cond \phi, y) \in \mathcal{P} \; \text{for all} \; p(\cdot \cond \param, \phi) \in \mathcal{F} \; \text{and} \; p(\cdot \cond \phi) \in \mathcal{P} \]